\documentclass[12pt]{report}
\usepackage[utf8]{inputenc}
\usepackage{amsmath}

\begin{document}

\chapter*{Riassunto Dynamic ARP Inspection Open Source}
\addcontentsline{toc}{chapter}{Riassunto}

Il presente elaborato di tesi affronta una delle criticità più persistenti e pervasive nella sicurezza delle reti locali (LAN): la vulnerabilità intrinseca dell'\textbf{Address Resolution Protocol (ARP)}. Progettato in un'epoca in cui la fiducia tra i nodi di rete era un assunto implicito, ARP manca di meccanismi nativi di autenticazione, esponendo le infrastrutture moderne a rischi severi quali l'\textbf{ARP Cache Poisoning} e i conseguenti attacchi \textbf{Man-in-the-Middle (MITM)}. Sebbene l'industria del networking abbia risposto con soluzioni efficaci, in primis la \textit{Dynamic ARP Inspection} (DAI) integrata negli apparati di fascia Enterprise (es. Cisco Catalyst), l'accesso a tali tecnologie rimane precluso a una vasta porzione di utenza a causa degli elevati costi di licenza e hardware.

L'obiettivo primario di questo lavoro è colmare tale divario tecnologico ed economico, proponendo un cambio di paradigma: la democratizzazione della sicurezza di livello 2 attraverso lo sviluppo di una soluzione \textbf{Dynamic ARP Inspection Open Source}. 
Come discusso approfonditamente nel capitolo 2, e in particolare nella sezione 2.4, la sicurezza non deve essere un privilegio accessorio riservato alle grandi infrastrutture aziendali, ma un requisito fondamentale integrabile in dispositivi eterogenei e accessibili. Questo progetto dimostra come sia possibile replicare, e potenzialmente estendere, le logiche di validazione delle soluzioni proprietarie utilizzando software aperto, trasparente e verificabile, svincolando l'amministratore di rete dal \textbf{vendor lock-in} e restituendo il controllo sull'integrità del traffico locale.

Sotto il profilo ingegneristico, la tesi descrive l'intero ciclo di vita del software, dalla progettazione architetturale all'implementazione in linguaggio \textbf{C}. La scelta del linguaggio e l'adozione di librerie standard come \textbf{Libpcap} e \textbf{POSIX Threads} rispondono a requisiti stringenti di performance e portabilità. L'architettura del daemon sviluppato si basa sul pattern concorrente \textbf{Produttore-Consumatore}, una scelta progettuale critica per disaccoppiare l'acquisizione dei pacchetti ad alta frequenza dalla logica di validazione.
Il sistema opera intercettando selettivamente le trame ARP Reply in transito sul gateway, sottoponendole a una verifica rigorosa: validazione di presenza nella tabella dei lease DHCP. Implementando una politica di sicurezza \textbf{Deny-by-Default}, il software considera qualsiasi messaggio ARP illegittimo fino a prova contraria, ovvero il presentare un'associazione $\langle \mbox{IP, MAC} \rangle$ presente nelle Lease DHCP. Particolare enfasi è stata posta sull'ottimizzazione delle strutture dati e sulla gestione della memoria. L'implementazione di una coda circolare (\textbf{Ring Buffer}) thread-safe e l'uso di primitive di sincronizzazione (\textbf{Mutex} e \textbf{Condition Variables}) hanno permesso di eliminare le attese attive, massimizzando l'efficienza della CPU. 

La validazione sperimentale, condotta in un ambiente virtualizzato complesso che simula topologie multi-LAN e scenari di attacco reali, ha prodotto evidenze quantitative significative. I test di carico hanno dimostrato come l'architettura multi-thread sia in grado di sostenere flood di oltre \textbf{34.000 PPS} mantenendo una latenza media di elaborazione nell'ordine dei microsecondi (\textbf{13-15 $\boldsymbol{\mu}$s})), un miglioramento di due ordini di grandezza rispetto alle implementazioni sequenziali.
Inoltre, il sistema ha dimostrato un'eccellente resilienza: anche in condizioni di \textbf{Denial of Service} su un segmento di rete, il traffico legittimo sugli altri segmenti viene preservato e processato senza perdite, confermando la robustezza della logica di isolamento dei flussi.

In conclusione, questo lavoro trascende la mera implementazione tecnica per delineare una strategia concreta di democratizzazione della sicurezza. I risultati empirici confermano il raggiungimento di prestazioni \textbf{Real-Time} su hardware non professionale, colmando l'attuale divario tra le costose soluzioni Enterprise e i dispositivi consumer intrinsecamente vulnerabili. La compatibilità nativa con l'ecosistema Linux Embedded, e in particolare la sinergia con la gestione dei lease di \textbf{Dnsmasq}, elegge piattaforme comunitarie come \textbf{OpenWrt} a servizio del futuro del progetto. Tale integrazione permetterebbe di estendere una protezione avanzata di livello 2 a contesti critici ma spesso privi di budget dedicato, quali le \textbf{Piccole e Medie Imprese (PMI)}, gli ambienti domestici evoluti (\textbf{Home Lab} e \textbf{Smart Working}) e le \textbf{Reti Civiche}, concretizzando la visione di:
\begin{center}
\textit{"Una sicurezza informatica non più intesa come privilegio economico, ma come bene comune accessibile e a servizio degli utenti finali"}.
\end{center}

\end{document}
