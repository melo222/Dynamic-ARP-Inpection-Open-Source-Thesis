\chapter{Introduzione}
\label{cap:introduzione}

Il presente elaborato si pone l'obiettivo di analizzare le criticità di sicurezza nelle reti locali, con particolare riferimento alle vulnerabilità del protocollo ARP, e di proporre una soluzione difensiva open source.

La trattazione è organizzata secondo un percorso logico che parte dai fondamenti teorici delle reti di calcolatori, indispensabili per comprendere il contesto operativo. Successivamente, viene discusso lo stato dell'arte, esaminando le soluzioni di sicurezza già presenti in letteratura e identificandone i limiti che hanno motivato lo sviluppo di una nuova implementazione.

Il cuore del lavoro verte sulla progettazione e lo sviluppo di un'architettura software inedita, di cui vengono dettagliate le scelte tecniche e le strategie di ottimizzazione. In conclusione, vengono presentate le evidenze sperimentali raccolte durante la fase di testing e delineate le possibili evoluzioni future del progetto.

    \section{Reti di Elaboratori, livello Fisico e Data Link del modello ISO/OSI}
    Le reti di elaboratori sono costituite da innumerevoli protocolli e dispositivi, la cui unione garantisce una trasmissione continua di dati attraverso molteplici client, server e reti. Lo schema logico più rappresentativo con il quale si riesce a modellare al meglio i principali protocolli e dispositivi di rete è il modello ISO/OSI, che prevede 7 livelli a disposizione per associare protocolli come IP, Ethernet, TCP, HTTP, etc.    

    I primi tre livelli, in particolare, ovvero il livello Fisico, Data Link e Rete, sono le fondamenta indispensabili per la connessione tra elaboratori; essi stabiliscono le regole di ingaggio tra dispositivi di rete fisici, gli indirizzi di comunicazione con cui individuare gli utenti, tra cui MAC e IP, e la gestione di reti multiple con gli indirizzi IP. 

    Nel dettaglio:
    \begin{itemize}
        \item Al livello fisico appartengono le convenzioni di comunicazione fisica tramite cavi in rame o fibra ottica, le conversioni da segnale elettrico ad analogico e gli standard di voltaggio per comunicare dati sotto forma di bit.
        \item Al secondo livello di Data Link risiedono protocolli, tra cui:
        \begin{itemize}
            \item MAC - Media Access Control            
            \item LLC - Logical Link Control (IEEE 802.2)
            \item Ethernet (IEEE 802.3) 
            \item Wi-Fi (IEEE 802.11)
        \end{itemize}
    \end{itemize}
    
    \begin{figure}[htbp]
        \centering
        \includegraphics[width=0.6\linewidth]{immagini/OSI_Model_Italiano.jpg} 
        \caption[Modello ISO/OSI]{Rappresentazione dei 7 livelli del modello ISO/OSI. (Fonte: Wikimedia Commons, CC BY-SA)}
        \label{fig:osi_model}
    \end{figure}
    Il primo protocollo regola l'accesso al mezzo fisico trasmissivo tramite il quale più schede di rete possono comunicare usando l'omonimo indirizzo MAC, composto da 48 bit o 12 cifre esadecimali univocamente identificabili (es. 00:1A:2B:3C:4D:5E).
    L'indirizzo MAC è usato come indirizzo di origine e di destinazione all'interno del frame del Data Link. Lo switch legge l'indirizzo MAC di destinazione per sapere esattamente su quale porta inoltrare il frame, garantendo che i dati vengano consegnati solo al dispositivo corretto sulla rete locale.   

    Il secondo livello, a differenza del MAC, si concentra maggiormente sul collegamento logico tra due dispositivi, su come stabilirlo e mantenerlo. Infatti, viene considerata la metà superiore del livello Data Link, in quanto fornisce funzionalità utili al livello sovrastante di Rete. 
    Il multiplexing è una di queste: la possibilità di distinguere pacchetti di livello 3 appartenenti a diversi protocolli nello stesso campo dati di una trama Data Link, come ad esempio IP e IPX. È opportuno notare che ormai gran parte del traffico è incapsulato nelle trame utilizzando IP a livello 3 (per esempio, anche lo stesso IPSec è un pacchetto IP).

    Ethernet e Wi-Fi sono, rispettivamente, due versioni del protocollo MAC su mezzo fisico cablato e wireless.

    Ora, tenendo sempre ben presente i livelli dello stack ISO/OSI, verranno descritti i protocolli protagonisti del livello di Rete e gli intermediari a cavallo tra esso e il livello Data Link. 

    \section{Livello di Rete e il ruolo del protocollo ARP}
    Il livello di Rete è responsabile della comunicazione tra reti geograficamente e logicamente differenti, seguendo il percorso non solo dal mittente al destinatario di una rete locale, come il livello Data Link, ma anche attraverso le molteplici reti e dispositivi intermedi Esso costituisce l'infrastruttura di rete, grazie principalmente al protocollo IP, Internet Protocol \cite{rfc791}, il quale assegna a ogni dispositivo un indirizzo di 32 bit arbitrario, in base alla configurazione statica o dinamica.   
    
    Un indirizzo, oltre al MAC, già di per sé univoco per ogni dispositivo di rete, nasce dalla necessità di una migliore scalabilità delle reti, in quanto l'indirizzo MAC è univoco, ma non gerarchico. Assumendo l'assenza di un'infrastruttura multi-reti appartenente al livello Tre, si avrebbe una gigantesca rete locale di livello Due con innumerevoli host e un ipotetico switch di livello due che permette loro di comunicare: ogni pacchetto inviato comporterebbe una ricerca nella tabella di enormi dimensioni contenente le proprie schede e il relativo MAC dell'host collegato, comportando inevitabilmente una notevole latenza ad ogni pacchetto. 
    Oltre al protocollo IP, spina dorsale di questo livello, sono presenti i protagonisti oggetto della discussione: DHCP e ARP.

        \subsection{DHCP}
        Il Dynamic Host Configuration Protocol \cite{rfc2131}, come suggerisce il nome, è il servizio di configurazione dinamica degli Host tramite una modalità di domanda-offerta che si conclude con l'attribuzione di indirizzi IP, ovvero un leasing di indirizzi con scadenza. Infatti, assumendo che sia abilitata l'assegnazione dinamica degli indirizzi IP, nel momento in cui un dispositivo si connette a una rete possiede solo il proprio MAC Address per comunicare e non ha accesso a una comunicazione superiore al livello 2. Inoltre, non conosce nemmeno gli indirizzi MAC degli altri dispositivi; dunque ha la necessità di individuarli per poter iniziare una comunicazione.
        
        Per completezza, bisogna specificare che il DHCP non risiede strettamente a livello 3, bensì è classificato come protocollo/servizio di livello Applicazione. Ciò è dovuto al fatto che fornisce un servizio di configurazione di rete utile alle applicazioni, o più genericamente al sistema operativo, per connettersi alla rete. Utilizza pacchetti UDP, livello di Trasporto, e porte dedicate: 67 per la ricezione e 68 per l'invio al client.
        
        Il leasing di indirizzi IP è composto da quattro fasi, note con l'acronimo "DORA":
        \begin{itemize}
            \item DHCP Discover
            \item DHCP Offer
            \item DHCP Request
            \item DHCP Acknowledge
        \end{itemize}
        La prima azione effettuata dal client è chiedere la presenza di un server DHCP nella rete che possa aiutarlo ad acquisire un indirizzo IP. Viene inviato un pacchetto che ha come indirizzi IP e MAC destinatari il valore di broadcast, rispettivamente a 32 e 48 bit tutti composti da 1 (255.255.255.255 e FF:FF:FF:FF:FF:FF:FF); oltre ad essi, è presente l'indirizzo IP sorgente interamente impostato a zero e il MAC del mittente.
        
        A seguito del pacchetto DHCP Discover, avviene la ricezione da parte del server sulla porta apposita, l'elaborazione della risposta in funzione degli IP disponibili e il conseguente invio del pacchetto DHCP Offer: a livelli superiori al livello di Rete, ovvero nel corpo del pacchetto UDP, è presente l'IP da assegnare al client e le opzioni della lease, come il tempo di scadenza, la maschera di rete e l'indirizzo del Gateway e DNS. A livello Tre sono presenti l'IP mittente del server e l'IP di destinazione impostato con indirizzo di broadcast; d'altronde, il client non ne possiede ancora uno. A livello Due vengono trascritti il MAC del server e quello del client.
        
        Il client riceve una o più offerte dai server DHCP, dato che è possibile avere un numero arbitrario di server DHCP in rete. Scegliere quale offerta ha maggiore priorità dipende principalmente dall'efficienza. In genere, si predilige la prima offerta che viene validata, eseguendo un controllo che i campi del messaggio siano corretti e plausibili.

        Scelta l'offerta, viene inviato il pacchetto DHCP Acknowledge per confermare al server DHCP la scelta e rendere ufficiale il lease. Da questo punto in poi, il client ha accesso al livello di Rete, può utilizzare IP e tutti i servizi che si basano su di esso: inoltrare pacchetti in differenti reti, avere accesso a servizi di livello Applicazione, etc. 

        Conclusa la spiegazione riguardo al leasing degli indirizzi IP, segue un breve riepilogo del suo ruolo nella rete. Esso, oltre a inizializzare gli host per la comunicazione a livello Tre, gestisce in memoria una tabella che contiene tutti i lease effettuati con i parametri descritti in precedenza. Dato che solitamente il servizio viene eseguito su router, come nel caso di una LAN con un solo dispositivo di livello Tre, quest'ultimo ha modo di vedere tutto il traffico in transito nello stesso dispositivo dove viene eseguito il DHCP. Questa nota tornerà utile successivamente nelle sezioni riguardanti ARP e i problemi di sicurezza di quest'ultimo.
        
        \subsection{ARP}
        L'Address Resolution Protocol (ARP) \cite{rfc826} è un protocollo di mappatura dinamica fondamentale che opera come ponte tra il livello di Rete e il livello Data Link all'interno di una rete locale. La sua esistenza è resa necessaria dal fatto che, sebbene i pacchetti IP contengano un indirizzo logico universale per l'instradamento globale, la consegna fisica effettiva di un frame a un nodo adiacente richiede l'indirizzo hardware specifico, ovvero il MAC address. ARP risolve il problema di trovare l'indirizzo MAC di destinazione quando si conosce solo il suo indirizzo IP. Ad esempio, tramite DHCP viene fornito  l'indirizzo del Gateway: 192.168.1.1. Non conoscendo il MAC, non si è in grado di realizzare un frame di livello 2 che possa incapsulare l'IP; da qui la necessità di chiedere a quale MAC appartenga un certo indirizzo IP.

        Il funzionamento di ARP si basa su un meccanismo di richiesta e risposta, formalmente ARP Request e ARP Reply. Quando un dispositivo desidera comunicare con un altro host sulla stessa rete locale conoscendo il suo indirizzo IP ma non il suo indirizzo MAC, il dispositivo mittente genera un messaggio ARP Request. Questo messaggio incapsula l'indirizzo IP del destinatario richiesto e viene inviato in broadcasting a tutti i nodi della rete locale. A Livello 2, il frame che trasporta la richiesta ARP ha l'indirizzo MAC di destinazione impostato su un valore di broadcast, garantendo che ogni dispositivo sulla LAN riceva e processi la richiesta.

        Tutti i dispositivi ricevono la richiesta ARP e, leggendo il campo che contiene l'indirizzo IP richiesto, lo confrontano con il proprio indirizzo IP. Solo il dispositivo che riconosce l'indirizzo IP come proprio risponde. Questo dispositivo genera un messaggio ARP Reply, che è di tipo unicast, indirizzato specificamente all'indirizzo MAC del mittente originale della richiesta. Il messaggio di risposta contiene la mappatura desiderata: "L'indirizzo IP X.X.X.X corrisponde all'indirizzo MAC Y:Y:Y:Y:Y:Y".
        
        Una volta ricevuta la risposta ARP, il dispositivo mittente memorizza l'associazione IP-MAC in una memoria temporanea chiamata cache ARP, sostanzialmente una tabella simile alla tabella delle lease  DHCP. Questa cache è cruciale per l'efficienza: finché l'associazione rimane valida (cioè non scade), il dispositivo non ha bisogno di ripetere il processo ARP per lo stesso indirizzo IP, ma può inserire direttamente il MAC address di destinazione nel frame di Livello 2 ogni volta che deve inviare pacchetti al determinato host. La durata della validità delle voci nella cache è gestita da un timer; una volta scaduta la voce, il dispositivo dovrà eseguire una nuova richiesta ARP per confermare la mappatura.
        
        ARP è un protocollo di instradamento locale e non attraversa mai i router. Quando un dispositivo deve inviare traffico a un host che si trova su una rete diversa, non esegue l'ARP per l'host remoto. Invece, esegue l'ARP per il suo Gateway Predefinito (il router). Dopo aver ottenuto il MAC address del router, il dispositivo incapsula il pacchetto IP (che ha ancora l'IP di destinazione remoto) nel frame di Livello 2 indirizzato al MAC del router. Il router, a sua volta, estrae il pacchetto IP e ne determina il percorso successivo, ripetendo il processo ARP sulla rete successiva se necessario. Questo dimostra chiaramente il ruolo di ARP come meccanismo essenziale di risoluzione degli indirizzi per la consegna del "next hop" sulla LAN.
                
    \section{I problemi di sicurezza del protocollo ARP}
    L'Address Resolution Protocol è notoriamente conosciuto per essere stato sviluppato senza schemi di autenticazione o verifica dei messaggi; la sua semplicità lo rende completamente privo di controlli di sicurezza e causa diverse vulnerabilità.

    Il principale e più noto problema di sicurezza associato all'ARP è l'\textit{ARP Cache Poisoning} (o ARP Spoofing). Questa tecnica di attacco sfrutta una fondamentale debolezza del protocollo: il fatto che sia \textit{stateless}, ovvero privo di stato, e non richieda autenticazione. Ciò comporta che un dispositivo aggiorni la propria cache ARP basandosi su qualsiasi messaggio ARP Reply ricevuto, accettandolo incondizionatamente anche se non ha mai inviato una richiesta ARP iniziale.

    L'attacco consiste nell'invio di messaggi ARP Reply falsificati a uno o più dispositivi presenti nella rete locale. Il funzionamento pratico è illustrato in dettaglio nella Figura \ref{fig:ARP_Cache_Poisoning}, che descrive l'evoluzione temporale dell'attacco attraverso tre fasi distinte.

    \begin{figure}[htbp]
        \centering
        \includegraphics[width=\textwidth, height=0.929\textheight, keepaspectratio]{immagini/ARP Cache Poisoning.jpg}
        \caption{Raffigurazione dell'attacco ARP Cache Poisoning: dal funzionamento normale (in alto) all'intercettazione MITM (in basso).}
        \label{fig:ARP_Cache_Poisoning}
    \end{figure}

    \subsection*{Fase 1: Funzionamento Normale}
    Nella parte superiore della figura è rappresentato lo stato legittimo della rete. Lo \textit{User 1} (192.168.1.10) comunica con lo \textit{User 2} (192.168.1.20) inviando pacchetti indirizzati al MAC address corretto del destinatario (\texttt{BB:...:BB}). Il traffico attraversa lo switch e raggiunge direttamente la destinazione prevista senza interferenze.

    \subsection*{Fase 2: Avvelenamento (Poisoning)}
    Nella parte centrale della figura avviene l'iniezione dei pacchetti malevoli. L'attaccante (\textit{HACKER}, 192.168.1.30) invia messaggi ARP Reply contraffatti a entrambe le vittime per corrompere le loro tabelle ARP:
    \begin{itemize}
        \item Allo \textit{User 1} viene comunicato che l'IP dello \textit{User 2} è associato al MAC dell'attaccante (\texttt{CC:...:CC}).
        \item Allo \textit{User 2} viene comunicato che l'IP dello \textit{User 1} è associato al MAC dell'attaccante (\texttt{CC:...:CC}).
    \end{itemize}

    \subsection*{Fase 3: Man-in-the-Middle (MITM)}
    La conseguenza dell'avvelenamento è visibile nella parte inferiore della figura. Quando lo \textit{User 1} tenta di inviare un nuovo pacchetto allo \textit{User 2}, utilizza l'associazione errata presente nella sua cache. Il frame ethernet viene quindi indirizzato al MAC \texttt{CC:...:CC} (come evidenziato nel pacchetto verde in basso a sinistra), portando lo switch a consegnare il dato all'attaccante anziché al destinatario reale.

    L'attaccante riceve il dato, può analizzarlo o alterarlo, e successivamente lo inoltra al destinatario legittimo (pacchetto rosso verso \textit{User 2}), rendendo l'attacco trasparente agli occhi delle vittime che credono di comunicare direttamente tra loro.

    Il verso della comunicazione può essere anche invertito con il traffico da \textit{User 2} a \textit{User 1}.